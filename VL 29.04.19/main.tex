\documentclass[12pt]{article}
\usepackage[utf8]{inputenc}
\usepackage[ngerman]{babel}
\usepackage{amsmath}
\usepackage{amsfonts}
\usepackage{amssymb}
\usepackage{graphicx}
\usepackage[left=2cm,right=2cm,top=2cm,bottom=2cm]{geometry}
\author{Bennet Bleßmann}
\begin{document}
$([Int], Maybe Char, [a])$



$length [] = 0$

$length (x:xs) = 1 + length xs$

$funsum xs ys = length xs + length ys$

\paragraph{}

\underline{Typannahme:}

$0,1::Int$

$(+)::Int->Int->Int$

$[] :: \forall a.[a]$

$(:) :: \forall a. a->[a]->[a]$

$length::\forall a.[a]->Int$

$funsum::[a]->[b]->Int$

$xs::[a]$

$ys::[b]$

$\frac{\frac{\frac{}{funsum::[a]->[b]->Int},\frac{}{xs::[a]}}{funsum\, xs ::[b] -> Int},\frac{}{ys :: [b]}}{funsum\, xs\, ys :: Int}$

\paragraph{}

\underline{Typinferenz:} möglichst allgemeiner Typausdrücke raten, sodass typkorrekt

$\hookrightarrow$ Danas/Milna PoPL'82

Idee: statt Typ raten: setze Typvariable ein, formuliere Bedingungen für diese $\approx$ Typgleichungen, löse das Gleichungssystem $\rightarrow$ allgemeinster Typ

\begin{enumerate}

\item \underline{Variablenumbenennung}, sodass verschiedene Regeln verschiedene Variablen haben

		$app\, []\, x = x				\rightarrow \{x 	\mapsto z\}		app\, []\, z = z$ \linebreak
		
		$app\, (x:xs)\, ys = x: app\, xs\, ys$
		
\item Bilde \underline{Ausdrucks/Typ-Paare}: Regel $l = r \rightsquigarrow l::a, R::a (a neue Typvariable)$
												$l|c = r \rightsquigarrow l::a,r::a, c::Bool (a neue Typvariable)$
												
\item Vereinfache Ausdrucks/Typ-Paare:
	Ersetze:
	\begin{itemize}
		\item $(e_1\, e_2)::\tau $ durch $e_1::a \rightarrow \tau, e_2::a $ (a neue Typvariable)

		\item $if \, e_1\, then\, e_2 \, else\, e_3$ durch $e_1::Bool, e_2::\tau, e_3::\tau$
		
		\item $e_1 \circ e_2 ::\tau$ durch $e_1::a , e_2 ::b , \circ::a->b->\tau$ (a,b neue Typvariablen)
		
		\item $\setminus x \rightarrow e ::\tau$ durch $x::a, e::b$ und Typgleichung $\tau \dot{=} a->b$ (a,b neue Typvariablen)
		
		\item $f::\tau$, f vordefiniert mit Typschema $\forall a_1,...,a_n.\tau^\prime$ durch $\tau \dot{=} \sigma(\tau^\prime)$ und $\sigma=\{a_1\mapsto b_1, ..., a_n \mapsto b_n\}$ ($b_1,...,b_n$ neue Typvariablen)	
	\end{itemize}
	
\item Nun: Paare der From $x::\tau$ (x Bezeichnung) setze alle Typen eines Bezeichners gleich: $x::\tau, x::\tau^\prime \rightsquigarrow \tau \dot{=} \tau^\prime$

\item Löse das Typgleichungssystem, d.h. finde allg. Unifikator $\sigma$, für $x::\tau$, $x::\sigma(\tau)$ der allgemeinste Typ von $x$
	Somit: Für typkorrekte Programme existiert immer ein allgemeinster Typ!
	
\end{enumerate}

Example for 4.
\begin{verbatim}

(twice f) x :: a 						f(f x)::a
twice f :: b \rightarrow a 	x ::b		f::d \rightarrow a  	f x ::d
twice::c \rightarrow (b \rightarrow a) 	f::c					f::e \rightawwor d  	x::e

b=e
c=d\rightarrow
c=e\rightarrow d
\end{verbatim}

Def.: - Substitution $\sigma$ \underline{Unifikator} für Gleichungssystem E falls $\forall l \dot{=} r \in E$ gilt $\sigma(l)=sigma(r)$
	- Subst. $\sigma$ \underline{allgem. Unifikator}(mgu) für E falls $\forall$ Unifi $\sigma^\prime$ für E exist. Subst $\varphi$ mit $\sigma^\prime \varphi \circ \sigma$
	
Satz(Robinson'): Falls Unifikator existiert, existiert auch mgu, der effektiv berechenbar ist.
Berechnung nach Martelli/Montanavi'82: Transformationsregel $\frac{A}{B}\downarrow$

De composition $\frac{\{k s_1 ... s_n \dot{=}k t_1 ... t_n \}\cup E}{\{s_1 \dot{=} t_1, ..., s_n \dot{=}t_n\} \cup E}$

Clash $\frac{\{h s_1 ... s_n \dot{=} h^\prime t_1 ... t_n\}}{Fail} h \neq h^\prime$

Elimination	$\frac{\{x \dot{=} x\} \cup E}{E}$

Swap $\frac{\{k s_1 ... s_n \dot{=} x \} \cup E}{\{x \dot{=} k s_1 ... s_n \}\cup E}$

Replace $\frac{\{x \dot{=} \tau\}\cup E}{ \{x \dot{=} \tau\}\cup \sigma(E)}$ x Typvar, $x$ kommt nicht in $\tau$ vor, $\sigma = \{x \mapsto \tau\}$

Occur Check $\frac{\{x\dot{=}\tau\} \cup E}{Fail}$ x Typvar, $x+\tau$, x kommt in $\tau$ vor

\paragraph{}
\underline{Typinferenz} für beliebig viele (rekursive) Funktionen
\begin{enumerate}
\item Sortiere Funktion nach Abhängigkeiten
\item Inferiere Typ der Basisfunktionen falls typkorrekt, dann dafür Typschemata
\item Inferiere Typ der darauf aufbauenden Funktionen, ... , wiederhohle diesen schrit bis fertig.
\end{enumerate}

\end{document}